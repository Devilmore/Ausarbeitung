%%%%%%%%%%%%%%%%%%%%%%%%%%%%%%%%%%%%%%%%%%%%%%%%%%%%%%%%%%%%%%%%%%%%%
%%%%%%%%%%%%%%%%%%%%%%%%%%%%%%%%%%%%%%%%%%%%%%%%%%%%%%%%%%%%%%%%%%%%
%% Template for writing dissertations with ThesisDIFCTUNL
%%
%% Version 20120920 (Sep 2012)
%% Departamento de Informática
%% Faculdade de Ciências e Tecnologia
%% Universidade Nova de Lisboa
%%
%% Project web page at: http://code.google.com/p/thesisdifctunl/
%%
%% BUGS and SUGGESTIONS: please submit an issue at the project web page
%%
%% Authors / Contributors:
%%     - João Lourenço (joao.lourenco@fct.unl.pt)
%%%%%%%%%%%%%%%%%%%%%%%%%%%%%%%%%%%%%%%%%%%%%%%%%%%%%%%%%%%%%%%%%%%%%
%%%%%%%%%%%%%%%%%%%%%%%%%%%%%%%%%%%%%%%%%%%%%%%%%%%%%%%%%%%%%%%%%%%%%
%% Main options (please read the manual for further details)
%% Options marked with (*) are the values by default
%% See also file "defaults.tex"
%%
%% - Class of text
%%           bsc		- BSc graduation report
%%           prepmsc	- Preparation of MSc dissertation
%%           (*)msc		- MSc dissertation
%%           propphd	- Proposal of PhD thesis
%%           prepphd 	- Preparation of PhD thesis
%%           phd		- PhD thesis
%% - Languages
%%           pt			- português
%%           (*)en		- english
%% - Single/double sided printing
%%           oneside	- single sided printing
%%           (*)twoside	- double sided printing
%% - Font size
%%           (*)11pt	- Use font size 11
%%           12pt		- Use font size 12
%% - Text encoding
%%           latin1		- Windows
%%           (*)utf8	- other systems (Linux, Mac)
%% - Bibliography
%%           bibbychapter	- Print the bibliography/references per chapter
%%			
%%
%% DON'T MIX TEXT ENCODINGS. All the files are saved using UTF8. 
%% If you want to use latin1, you should first re-save all these
%% files (*.tex, *.cls,  ...) using latin1 encoding first
%%
%%%%%%%%%%%%%%%%%%%%%%%%%%%%%%%%%%%%%%%%%%%%%%%%%%%%%%%%%%%%%%%%%%%%%
%%%%%%%%%%%%%%%%%%%%%%%%%%%%%%%%%%%%%%%%%%%%%%%%%%%%%%%%%%%%%%%%%%%%%
	%\documentclass[bsc,en,oneside,12pt,utf8,a4paper,]{thesisdifctunl}
	\documentclass[m,bachelor,binding,oneside,palatino]{thesisdifctunl}



%%%%%%%%%%%%%%%%%%%%%%%%%%%%%%%%%%%%%%%%%%%%%%%%%%%%%%%%%%%%%%%%%%%%%
%%%%%%%%%%%%%%%%%%%%%%%%%%%%%%%%%%%%%%%%%%%%%%%%%%%%%%%%%%%%%%%%%%%%%
%%  BEGINING OF USER COSTUMIZATION
%%%%%%%%%%%%%%%%%%%%%%%%%%%%%%%%%%%%%%%%%%%%%%%%%%%%%%%%%%%%%%%%%%%%%
%%%%%%%%%%%%%%%%%%%%%%%%%%%%%%%%%%%%%%%%%%%%%%%%%%%%%%%%%%%%%%%%%%%%%
%====================================================================
% Additional package syou may ant to use (comment those not needed)
%====================================================================

% To enable customizble enumerates
\usepackage{paralist}
\usepackage{longtable}

\usepackage{amsmath}

% Beautiful simple tables
\usepackage{booktabs}

% Use colors in background of table cells
%\usepackage{colortbl}

% Sort citations - ONLY if using "plain" style, otherwise keep commented
\usepackage[sort]{cite}

% Fontification of source code listings
\usepackage{listings}

% To aggregate multiple figures in a single one with subfigures
\usepackage{subfigure}

\usepackage{wasysym}
\usepackage{float}
\restylefloat{table}

% Pseudo Code
\usepackage[ruled,vlined]{algorithm2e}
%\usepackage{algpseudocode}
%\usepackage{algorithm}

% To register TODO notes in the text
\usepackage[textsize=footnotesize]{todonotes}
\setlength{\marginparwidth}{1.5cm}

\usepackage{ifthen}
\let\oldcite=\cite
\renewcommand\cite[1]{\ifthenelse{\equal{#1}{NEEDED}}{[citation~needed]}{\oldcite{#1}}}
\usepackage[nottoc,numbib]{tocbibind}

% To have text wrapping figures
% \usepackage{wrapfig}

%====================================================================
% Standard configuration for user included packages
%====================================================================

% Where to look for figures
\graphicspath{{Logo/}{Figures/}{Chapters/Figures/}} 

% Setup of listings, for more information check package manual
\lstset{
    captionpos=t,
    basicstyle={\ttfamily\footnotesize},
    numbers=left,
    numberstyle={\ttfamily\tiny},
    tabsize=2,
    language=Java,
    float,
    frame=single,
}

%%============================================================

% Title of the dissertation/thesis
% USe "\\" to break the title into two or more lines
\title{Local Area Navigation for Multi-axle Vehicles using Machine Learning Algorithms}
\author[m]{Andreas Barthen
\degreecourse{Informatik}


%% old shit
% Author
\authordegree{}

% Date
\themonth{March}
\theyear{2013}


% Supervisors (maximum of 9)
% use [f] for female and [m] for male
\adviser[m]{Christian Schwarz}{Dipl. Informatiker}{\\&Universität Koblenz-Landau}
\adviser[m]{Ulrich Furbach}{Prof.\ Dr.}{\\&Universität Koblenz-Landau}

\acknowledgementsfile{acknowledgements}

% Quote
% \quotefile{quote}

% Resume/summary text in German
\deabstractfile{zusammenfassung}

% Resume/summary text in English
\enabstractfile{abstract}

% The Table of Contents is always printed. The other lists may be omited.


% The List of Figures. Comment to omit.
\printlistoffigures 

% The List of Tables. Comment to omit.
\printlistoftables 

% The List of Code Listings. Comment to omit.
%\printlistoflistings 


% Text chapters
% syntax: \chapterfile{file}
\chapterfile{chapter1}
\chapterfile{chapter3}
\chapterfile{chapter2}
%\chapterfile{chapter4}
\chapterfile{chapter6}
\chapterfile{chapter5}
\chapterfile{chapter7}
\chapterfile{chapter8}



% Text appendixes
% sintax: \appendixfile{file}
% Comment if not needed
\appendixfile{appendix1}

% BibTeX bibiography files. Can be used multiple times with a single file nae each time.
\addbibfile{bibliography}


%%============================================================
%%
%%  END OF USER COSTUMIZAITON
%%
%%============================================================
%%============================================================
%% Please do not change below this point!!! :)
%%============================================================
%% Begining of document
\begin{document}

%%------------------------------------------------------------
%% Before main text
\frontmatter

% The first front page
%\frontpage
\maketitle

% Print the copyright page for final versions	
%\printcopyright

% Dedicatory will only be printed if adequate for the document type
% \printdedicatory

% Acknowledgments will only be printed if adequate for the document type
\printacknowledgements

% Print the quote if file exists
%\printquote

% Abstracts/resumes/summaries in two languages. The first abstract will
% use the document main language and second the foreign language
\printabstract

%%------------------------------------------------------------
% Tables/lists of contents
\tableofcontents 

% Always print the table of contents
\printotherlists 

% Print other lists of contents according to instructions given above
%%------------------------------------------------------------
% Print document chapters
\printchapthers

%%------------------------------------------------------------
% Print the bibliography
\printbib

%%------------------------------------------------------------
% Print appendixes, if any!
\printappendixes 


%% -----------------------------------------------------------
% Bibliography here
\begin{thebibliography}{9}

\bibitem{1}
  J.T. Schwartz and M. Sharir,
  "`A survey of motion planning and related geometric algorithms"',
  Artificial Intelligence 37, 157–169
  1988.
	
\bibitem{2}
	N.J. Nilsson,
	“A mobile automaton: an application of artificial intelligence techniques”,
	Proc. of the 1st Int. Joint Conf. on Artificial Intelligence, pp. 509–520,
	Washington D.C.
	1969

\bibitem{3}
	H. Mitchell, 
	“An algorithmic approach to some problems in terrain navigation”,
	Artificial Intelligence 37, 171–201
	1988

\bibitem{4}
	C. O’Dunlaing and C.K. Yap,
	“A retraction method for planning the motion of a disc”,
	Journal of Algorithms 6, 104–111 
	1982

\bibitem{5}
	C. O’Dunlaing, M. Sharir and C.K. Yap,
	“Retraction: a new approach to motion planning”,
	Proc. of the 15th ACM Symp. on the Theory of Computing, pp. 207–220.
	Boston
	1983
	
\bibitem{6}
	E. W. Dijkstra,
	“A note on two problems in connection with graphs”,
	Numerische Math. 1, 269–271 
	1959
	
\bibitem{7}
	P.E. Hart, N.J. Nilsson and B. Raphael,
	“A formal basis for the heuristic determination of minimum cost paths”,
	IEEE Trans.on Systems, Man, and Cybernetics SMC-4, No. 2, 100–107
	July, 1968
	
\bibitem{8}
	Andreas C. Nearchou,
	"`Path planning of a mobile robot using genetic heuristics"',
	Robotica (1998) Volume 16, pp. 575-588,
	Cambridge University Press,
	Patras,
	1997	
	
\bibitem{9}
	Lavi M. Zamstein
	"`Koolio: Path planning using reinforcement learning on a real robot platform"'
	University of Florida
	2006
	
\bibitem{10}
	Smart, W. and Kaelbling, L.
	"`Effective Reinforcement Learning for Mobile Robots"',
	Proceedings of the 2002 IEEE International Conference on Robotics & Automatons,
	Washington, DC,
	2002
	
\bibitem{11}
	Indrani Goswami, Pradipta Kumar Des, Amit Konar, R. Janarthana,
	"`Extended Q-Learning Algorithm for Path-Planning of a Mobile Robot"',
	Simulated Evolution and Learning, Lecture Notes in Computer Science Volume 6457, pp 379-383
	2010

\bibitem{12}
	Christian Schwarz,
	"`Entwicklung eines Regelungsverfahrens zur Pfadverfolgung für ein Modellfahrzeug mit einachsigem Anhänger"',
	Diplomarbeit,
	Koblenz,
	2009
	
\bibitem{15}
	J.H. Holland,
	"`Adaptation in Natural and Artificial Systems"',
	University of Michigan Press,
	Ann Arbor, 
	1975
	
\bibitem{16}
	Thomas Bäck,
	"`Evolutionary Algorithms in Theory and Practice"', 106-120,
	Oxford University Press,
	1996
	
\bibitem{17}
	J.E. Baker
	"`Adaptive selection methods for genetic algorithms"',
	Proceedings of the 1st International Conference on Genetic Algorithms, 101-111
	1985
	
\bibitem{18}
	Richard A. Caruana, Larry J. Eshelman, J. David Schaffer,
	"`Representation and hidden bias II: eliminating defining length bias in genetic search via shuffle crossover"'
	Proceeding IJCAI'89 Proceedings of the 11th international joint conference on Artificial intelligence - Volume 1, 750-755,
	1989
	
\bibitem{19}
	Richard A. Caruana, Larry J. Eshelman, J. David Schaffer,
	"`Biases in the crossover landscape"',
	Proceedings of the third international conference on Genetic algorithms, 10-19
	1989
	
\bibitem{20}
	Richard A. Caruana, Larry J. Eshelman, J. David Schaffer, Rajarshi Das,
	"`A Study of Control Parameters Affecting Online Performance of Genetic Algorithms for Function Optimization"',
	Proceedings of the 3rd International Conference on Genetic Algorithms, 51-60
	1989
	
\bibitem{21}
	John .J. Grefenstette,
	"`A User's Guide to GENESIS"',
	Navy Center for Applied Research in Artificial Intelligence,
	Washington, D.C.,
	1987
	
\bibitem{22}
	Kenneth Alan De Jong,
	"`An analysis of the behavior of a class of genetic adaptive systems"',
	Doctoral Dissertation, 
	University of Michigan Ann Arbor, MI, USA,
	1975 
	
\bibitem{23}
	J. David Schaffer,	Amy Morishima,
	"`An adaptive crossover distribution mechanism for genetic algorithms"',
	Proceedings of the Second International Conference on Genetic Algorithms on Genetic algorithms and their application, 36-40,
	1987

\bibitem{24}
	John .J. Grefenstette,
	"`Optimization of Control Parameters for Genetic Algorithms"'
	IEEE Transactions on Systems, Man and Cybernetics, Volume 16, Issue 1, 122-128
	1986
	
\bibitem{25}
	D.E. Goldberg, C.L. Bridges
	"`An analysis of a reordering operator on a GA-hard problem"',
	Biological Cybernetics, Volume 62, Issue 5, 397-405
	1990
	
\bibitem{26}
	D.E. Goldberg,
	"`Genetic Algorithms in Search, Optimization, and Machine Learning"',
	Addison-Wesley, Reading, Mass.,
	1989
	
\bibitem{27}
	M. Gerdts,
	"`The single track model"',
	Universität Bayreuth,
	2003
	
\bibitem{28}
	C. Altafini,
	"`Some properties of the general n-trailer"',
	International Journal of Control 74,
	2001
	
\bibitem{29}
	Matthew T. Mason
	"`Nonholonomic constraint"',
	Mechanis of Manipulation,
	2013
	
\bibitem{30}
	Steven M. LaValle,
	"`Planning Algorithms"',
	Cambridge University Press,
	2006
	
\bibitem{31}
	C. Pradalier, K. Usher,
	"`Robust trajectory tracking for a reversing tractor trailer"',
	In: Journal of Field Robotics 25, Nr. 6, 378-399,
	2008
	
\bibitem{32}
	A. Preißer,
	"`Heuristische Ansätze zur Regelung eines Fahrzeugs mit Anhänger"',
	Diplomarbeit, Universität Koblenz-Landau,
	2001
	
\bibitem{33}
	Wikipedia contributors,
	"`Rapidly exploring random tree"',
	Wikipedia, The Free Encyclopedia,
	2013
	
\bibitem{34}
	Steve LaValle,
	Rapidly-exploring Random Trees (RRTs),
	The RRT Page

\bibitem{35}
	Stephan Wirt
	"`Autonome gründliche Exploration unbekannter Innenräume mit dem mobilen Roboter „Robbie“"',
	Studienarbeit im Fach Computervisualistik,
	2007
	
\bibitem{36}
	The Open Toolkit library,
	http://www.opentk.com/project/license,
	2006-2009
	
\bibitem{37}
	Dieter Zöbel
	"`Autonomous Driving in Goods Transport"'
	
\bibitem{38}
	Cat Minestar System,\\
	http://www.catminestarsystem.com/capability\_sets/command/packages/autonomous
	
\bibitem{39}
	J. Hopcroft, J.T. Schwartz, M. Sharir,
	"`On the complexity of motion planning for multiple independent objects; PSPACE-hardness of the warehouseman's problem"',
	International Journal of Robotics Research, 3(4):76-88,
	1984.
	
\bibitem{40}
	James J. Kuffner, Jr., Steven M. LaValle,
	"`RRT-Connect: An Efficient Approach to Single-Query Path Planning"',
	In Proc. 2000 IEEE Int’l Conf. on Robotics and Automation (ICRA 2000)

\end{thebibliography}

%% End of Bibliography

%% End of document
\end{document} 
