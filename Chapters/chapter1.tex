\chapter{Introduction}
\label{cha:introduction}

Navigation systems are a common occurence and present in almost any vehicle any smartphone, that is especially true for professional vehicles like trucks. However the directions provided by a navigation system end as soon as the road end, so when the truck enters the transit station the driver is on his own. With many trucks in the same areaone might want to control the routes they take to make sure that every truck gets to it's position safely and quickly. What is required for this is not navigation but path-planning, which is a much more complex problem, especially with multi-axle vehicles like trucks with one, or maybe more, trailers. Finding such paths is possible with common algorithms, which will also be presented in the first half of this paper, these are however not ideal and optimizing them can take a great amount of time. The more complexity, in our case more trailers and more freedom to move, are added, the hrader it becomes to find an optimal path using only incremental algorithms. In order to solve find a more efficient solution to this problem a path-planning software has been developed, which is now described in this paper. The software is based an existing simulation and path-correction software developed by the AG Echtzeitsysteme at the University Koblenz-Landau and employs a Genetic Algorithm to generated an optimized path for general-n-trailers driving in reverse. 
To this end we will first cover a lot of basics, that is, common alternative algorithms working on the same problem, the simulation software used as well as some facts about the vehicle used in our simulations. The idea and development of the planning software is then described in detail in the second half of this paper and finally, the results are evaluated and compared to other solutions. For the purpose of this work we limit ourselves to only one trailer, though the software can theoretically hande arbitrarily many, and we assume the vehicle does not change directions, driving in reverse only. The consequences of lifting either of these restrictions will also be considered at the end of this work.
