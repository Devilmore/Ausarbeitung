\chapter{Introduction}
\label{cha:introduction}

Navigation systems are a common occurrence and present in almost any vehicle and any smartphone, that is especially true for professional vehicles like delivery trucks. However, the directions provided by a navigation system end as soon as the road does, so when the truck enters the transit station the driver is on his own. With many trucks in the same area, one might want to control the routes they take to make sure that every truck gets to its position safely and quickly. In the next step one might even want to automate the entire process by having the trucks drive autonomously within a defined area, in this case the transit station. Such an autonomous system would lower fuel consumption, safe time and extend the range of the vehicles used \cite{37}. It would also make operation safer, especially in dangerous environments such as mines where, in fact, fully autonomous trucks are already being used in everyday work \cite{38}. What is required for such autonomy is not navigation but path-planning, which is a much more complex problem, especially with multi-axle vehicles like trucks with one, or maybe more, trailers. For a completely autonomous transit system one would also require a central system scheduling the routes of all trucks to make sure everything arrives on time and no trucks crash into each other. This component however will not be considered here as we want to concentrate on the path-planning task, with a focus on trucks with several trailers and the possibility to drive such complex vehicles backwards. Finding paths for such vehicles is possible with common algorithms, which will also be presented in the first half of this thesis, these are however not ideal and optimizing them can take a great amount of time. The greater the complexity, in our case the more trailers and more freedom to move, are added, the harder it becomes to find an optimal path using only incremental algorithms.\\
In order to find a more efficient solution to this problem a machine-learning based path-planning software has been developed, which is now described in this paper. The software is based on an existing simulation and path-correction software developed by the AG Echtzeitsysteme at the University of Koblenz-Landau and employs a genetic algorithm to generate an optimized path for general-n-trailers.\\
The software should be able to plan paths for a given map and a given vehicle without its performance suffering too much from a higher trailer count, which is why we chose to approach the problem with a genetic algorithm, a proven \cite{9} solution to complex path-planning tasks. We also want to be able to quickly change our map and vehicle to adjust to new planning tasks and be able to configure parameters within the algorithm to optimize our performance and make comparisons between different setups. \\
To this end we will first cover a lot of basics, that is, common alternative algorithms working on the same problem, the simulation software used, as well as some facts about the vehicle used in our simulations. The idea and development of the planning software is then described in detail in the second half of this paper and finally the results are evaluated in the last chapter. For the purpose of this work we limit ourselves to only one trailer, though the software can theoretically handle arbitrarily many, and we assume the vehicle does not change directions, driving in forward direction only. The consequences of lifting either of these restrictions will also be considered at the end of this work.\\
\\
The program can be downloaded at http://tinyurl.com/mx28p2l