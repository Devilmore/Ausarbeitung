\chapter{Conclusion and Future Work}
\label{cha:conclusion}

We have shown the usability of machine learning algorithms, with the genetic algorithm as an example, as a tool for finding or optimizing paths in high dimensional spaces under consideration of nonholonomic constraints. We have also introduced alternative algorithms and, as far as possible without actually implementing every possible algorithm under the same circumstances, made some comparisons to our own results. 
As already mentioned in \ref{cha:algorithm_details} many more alternative implementations and further optimizations of the genetic algorithm are possible, some of which will be summarized here.

The accuracy of the GA could be further optimized by adjusting the collision detection to avoid getting close to obstacles, not just avoiding the obstacles themselves. The overall quality of the path could be raised by considering more factors, such as length, number of turns or minimum distance to any obstacle along the path.
The current implementation is tested to provide the best performance with the given operators, however, since there are many different possible GA operators and many ways to weight and combine these, there are possibly combinations that would provide better results that have not been considered here. While there are certain operators that are more suited to a given task than other, many of these preferences can only be determined by extensive testing of all possibilities, which would go beyond the scope of this thesis.
The current version of the program makes two assumptions about the vehicle that may be lifted with slight alterations and further testing: The vehicle is always driving forward, never changing direction and the vehicle only has a maximum of one trailer. The software is already capable of handling an infinite number of trailers, however, since this has not been tested, it is currently not possible to define more than one trailer in the GUI. This assumption is made mostly due to the fact that general-n-trailers with infinite trailers are purely theoretical. As shown in \cite{8}, GAs are not slowed down by higher complexity as much as alternative solutions, so while the performance of our proposed algorithm would certainly suffer from having to consider more trailers, it would not slow down as much as iterative algorithms and would further outperform classic solutions. Our own testing in \ref{cha:evaluation} confirms this assumption.
Changing directions during the drive is partially implemented insofar that both the genome and the path part classes contain the information necessary, in the current version these are all set to default values however. The program has not been tested with such paths and the current random path generator does not allow for the path to contain corners since it always uses the direction of the end of the previous path part when generating the next one. This limitation would have to be loosened since a change in direction would obviously break this restriction.
The algorithm may also be optimized at its very basis, that is, the genome representation. The current representation limits both the maximum and minimum length of the path as well as the granularity of the choices made. By using different or simply more detailed representations the algorithm may be able to come up with better paths, gain better performance or both.

While many such optimizations are possible and are certainly already being worked on by various research groups around the world, the purpose of this paper was not to develop a best possible algorithm, but simply to show that such machine learning algorithms for path-planning tasks exist and are actually useful. While the given algorithm may be far from optimal, it is a working solution for high dimensional problem which, depending on the chosen complexity, can outperform classic path planning algorithms.