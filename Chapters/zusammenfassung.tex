\abstractDE 

Diese Arbeit beschreibt die Implementation eines Pfadplanungs-Algorithmus für Seriengespannfahrzeuge mithilfe von Maschinellen Lernalgorithmen. Zu diesem Zwecke wird ein allgemeiner Überblick über genetische Algorithmen gegeben, alternative Ansätze werden ebenfalls kurz erklärt. Die Software die zu diesem Zwecke entwickelt wurde basiert auf der EZSystem Simulationssoftware der AG Echtzeitsysteme der Universität Koblenz-Landau, sowie auf der von Christian Schwarz entwickelten Pfadkorrektursoftware, die ebenfalls hier beschrieben wird. Diese enthält auch eine Beschreibung des, zu Simulationszwecken, verwendeten Fahrzeugs. Genetische Algorithmen als Lösung von Pfadplanungsproblemen in komplexen Szenarien werden dann, basierend auf der entwickelten Simulationssoftware, evaluiert und diese Ergebnisse werden dann mit alternativen, nicht-maschinellen Lernalgorithmen, verglichen. Diese werden ebenfalls kurz erläutert.

% Deutsche Schlüsselwörter
%\begin{keywords}
%Palavras-chave (em português) \ldots
%\end{keywords}
